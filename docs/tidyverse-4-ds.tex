\documentclass[]{book}
\usepackage{lmodern}
\usepackage{amssymb,amsmath}
\usepackage{ifxetex,ifluatex}
\usepackage{fixltx2e} % provides \textsubscript
\ifnum 0\ifxetex 1\fi\ifluatex 1\fi=0 % if pdftex
  \usepackage[T1]{fontenc}
  \usepackage[utf8]{inputenc}
\else % if luatex or xelatex
  \ifxetex
    \usepackage{mathspec}
  \else
    \usepackage{fontspec}
  \fi
  \defaultfontfeatures{Ligatures=TeX,Scale=MatchLowercase}
\fi
% use upquote if available, for straight quotes in verbatim environments
\IfFileExists{upquote.sty}{\usepackage{upquote}}{}
% use microtype if available
\IfFileExists{microtype.sty}{%
\usepackage{microtype}
\UseMicrotypeSet[protrusion]{basicmath} % disable protrusion for tt fonts
}{}
\usepackage{hyperref}
\hypersetup{unicode=true,
            pdftitle={Tidyverse数据分析},
            pdfauthor={Xiaotao Shen(申小涛)},
            pdfborder={0 0 0},
            breaklinks=true}
\urlstyle{same}  % don't use monospace font for urls
\usepackage{natbib}
\bibliographystyle{apalike}
\usepackage{longtable,booktabs}
\usepackage{graphicx,grffile}
\makeatletter
\def\maxwidth{\ifdim\Gin@nat@width>\linewidth\linewidth\else\Gin@nat@width\fi}
\def\maxheight{\ifdim\Gin@nat@height>\textheight\textheight\else\Gin@nat@height\fi}
\makeatother
% Scale images if necessary, so that they will not overflow the page
% margins by default, and it is still possible to overwrite the defaults
% using explicit options in \includegraphics[width, height, ...]{}
\setkeys{Gin}{width=\maxwidth,height=\maxheight,keepaspectratio}
\IfFileExists{parskip.sty}{%
\usepackage{parskip}
}{% else
\setlength{\parindent}{0pt}
\setlength{\parskip}{6pt plus 2pt minus 1pt}
}
\setlength{\emergencystretch}{3em}  % prevent overfull lines
\providecommand{\tightlist}{%
  \setlength{\itemsep}{0pt}\setlength{\parskip}{0pt}}
\setcounter{secnumdepth}{5}
% Redefines (sub)paragraphs to behave more like sections
\ifx\paragraph\undefined\else
\let\oldparagraph\paragraph
\renewcommand{\paragraph}[1]{\oldparagraph{#1}\mbox{}}
\fi
\ifx\subparagraph\undefined\else
\let\oldsubparagraph\subparagraph
\renewcommand{\subparagraph}[1]{\oldsubparagraph{#1}\mbox{}}
\fi

%%% Use protect on footnotes to avoid problems with footnotes in titles
\let\rmarkdownfootnote\footnote%
\def\footnote{\protect\rmarkdownfootnote}

%%% Change title format to be more compact
\usepackage{titling}

% Create subtitle command for use in maketitle
\providecommand{\subtitle}[1]{
  \posttitle{
    \begin{center}\large#1\end{center}
    }
}

\setlength{\droptitle}{-2em}

  \title{Tidyverse数据分析}
    \pretitle{\vspace{\droptitle}\centering\huge}
  \posttitle{\par}
    \author{Xiaotao Shen(申小涛)}
    \preauthor{\centering\large\emph}
  \postauthor{\par}
      \predate{\centering\large\emph}
  \postdate{\par}
    \date{2019-11-08}

\usepackage{booktabs}

\begin{document}
\maketitle

{
\setcounter{tocdepth}{1}
\tableofcontents
}
\hypertarget{ux524dux8a00}{%
\chapter{前言}\label{ux524dux8a00}}

一直准备详细的把自己学习\texttt{tidyverse}用于数据分析的所有知识记录下来,原来准备使用博客,后来发现使用博客写书还是不太好用,所以还是准备使用\texttt{bookdown}.整本书准备以\texttt{R\ for\ data\ science},\texttt{tidyver}官网以及自己在实际处理数据过程中遇到的问题为基础进行记录.整体分几章还没有想好.争取每天晚上写一点.

示例数据尽量使用在R中的自带数据,这样也方便大家以及我自己重复.

\hypertarget{long-wide-data}{%
\chapter{长宽数据转换}\label{long-wide-data}}

长数据和宽数据是我们在数据分析过程中最经常遇到的两种数据类型.

\begin{itemize}
\tightlist
\item
  宽数据: 最为常见的应该是宽数据.我们可以举个例子.每一行为一个基因,每一列为一个样品,每一个cell是一个数值.这时候得到的数据框就是宽数据.
\end{itemize}

\begin{table}

\caption{\label{tab:unnamed-chunk-1}Demo of wide data}
\centering
\begin{tabular}[t]{lrrrrr}
\toprule
  & sample\_1 & sample\_2 & sample\_3 & sample\_4 & sample\_5\\
\midrule
gene\_1 & 83 & 48 & 89 & 80 & 51\\
gene\_2 & 78 & 88 & 39 & 99 & 56\\
gene\_3 & 75 & 53 & 9 & 15 & 36\\
gene\_4 & 41 & 44 & 27 & 40 & 13\\
gene\_5 & 82 & 79 & 12 & 54 & 47\\
\bottomrule
\end{tabular}
\end{table}

可以看到宽数据的并不直接.而且有点像是二维的数据,比如我想知道sample1的gene2的表达量是多少,那就需要我从第一列(sample\_1)的第二行(gene\_2)定位到那个cell,然后得到数值.但是如果我们的定性信息多余两个呢?比如我们现在的问题是某个样品在某种条件下的某个基因的表达量.那么用宽数据就很难进行展示.所以另外一个比较常见的就是长数据.

\begin{itemize}
\tightlist
\item
  顾名思义,长数据就比较长.对于长数据来说,每一行是定义了一个case,前面的列都是用来描述这个case的属性,比如用长数据来展示我们上面的宽数据,就可以表示为下面的表格.
\end{itemize}

\begin{table}

\caption{\label{tab:unnamed-chunk-2}Demo of long data}
\centering
\begin{tabular}[t]{llr}
\toprule
Sample\_name & Gene\_name & Expression\\
\midrule
sample\_1 & gene\_1 & 9\\
sample\_2 & gene\_1 & 87\\
sample\_3 & gene\_1 & 31\\
sample\_4 & gene\_1 & 80\\
sample\_5 & gene\_1 & 35\\
\addlinespace
sample\_1 & gene\_2 & 82\\
sample\_2 & gene\_2 & 100\\
sample\_3 & gene\_2 & 58\\
sample\_4 & gene\_2 & 30\\
sample\_5 & gene\_2 & 65\\
\addlinespace
sample\_1 & gene\_3 & 98\\
sample\_2 & gene\_3 & 18\\
sample\_3 & gene\_3 & 55\\
sample\_4 & gene\_3 & 38\\
sample\_5 & gene\_3 & 32\\
\addlinespace
sample\_1 & gene\_4 & 79\\
sample\_2 & gene\_4 & 73\\
sample\_3 & gene\_4 & 22\\
sample\_4 & gene\_4 & 75\\
sample\_5 & gene\_4 & 33\\
\addlinespace
sample\_1 & gene\_5 & 3\\
sample\_2 & gene\_5 & 34\\
sample\_3 & gene\_5 & 8\\
sample\_4 & gene\_5 & 51\\
sample\_5 & gene\_5 & 67\\
\bottomrule
\end{tabular}
\end{table}

从上面两个长款示例数据我们可以清晰的看到他们的区别.

\hypertarget{literature}{%
\chapter{Literature}\label{literature}}

Here is a review of existing methods.

\hypertarget{methods}{%
\chapter{Methods}\label{methods}}

We describe our methods in this chapter.

\hypertarget{applications}{%
\chapter{Applications}\label{applications}}

Some \emph{significant} applications are demonstrated in this chapter.

\hypertarget{example-one}{%
\section{Example one}\label{example-one}}

\hypertarget{example-two}{%
\section{Example two}\label{example-two}}

\hypertarget{final-words}{%
\chapter{Final Words}\label{final-words}}

We have finished a nice book.

\bibliography{book.bib,packages.bib}


\end{document}
